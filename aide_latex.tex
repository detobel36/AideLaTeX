\documentclass[a4paper,11pt]{article}
\usepackage[french]{babel}
\usepackage[utf8]{inputenc}
\usepackage[T1]{fontenc}
\usepackage{amsmath}
\usepackage{amsfonts}
\usepackage{amssymb}
\usepackage{hyperref}
\usepackage{graphicx}  % \includegraphics[scale=0.6]{image.png}\\
\usepackage[top=2cm, bottom=2cm, left=2cm, right=2cm]{geometry}
\usepackage{tikz}

\usepackage{fancyhdr}
\pagestyle{fancy}
\fancyhead[C]{\includegraphics[scale=0.2]{./logo.jpg}}
\fancyhead[L]{}
\fancyhead[R]{}


%opening
\title{Fiche d'aide \LaTeX}
\author{Rémy Detobel}

\begin{document}

\maketitle

\newpage
\tableofcontents
\newpage


% - Introduction ---------------------------------------------------------------------------------------------------------
\section{Introduction}
  \subsection{Les avantages}
    \LaTeX\ permet de créé un document PDF via des balises.  Il est très pratique pour écrire un rapport, des formules mathématiques, une table des matières...  Bref, ce qui (en général) est compliqué sur des éditeurs tel que Word devient ``simple'' ici (lorsque l'on a pris l'habitude de l'utiliser).  Sachez également que les profs aiment voir de jolie rapport en \LaTeX.
    
  \subsection{Les IDE}
    Les IDE viens de ``Integrated Development Environment``, ici on ne fait pas de développement...  On devrait dire ILE ''Integrated \LaTeX\ Environment.  Mais entre nous, je vous propose de garder IDE.\\
    Ces IDE vous permettent de créer un document \LaTeX\ plus facilement que via un simple éditeur de texte.  Votre éditeur vous permet d'ajouter une balise juste en poussant sur un bouton, fait de l'autocomplétion, ...  Bref, pleins de petites choses qui vous permettrons d'aller plus vite.\\
    Nombre de sondés : 9
    \begin{itemize}
      \item texWork : 2
      \item Kile : 1
      \item LateXila :1
      \item TexMaker : 1
      \item TexStudio : 2
      \item sublime texte : 1
      \item vim : 1
      \item notepad++ : 1
    \end{itemize}
    Les IDE intègres en général un compilateur permettant de créer le document PDF.  Si vous n'utilisez pas d'IDE vous pouvez tout de même écrire du \LaTeX\ dans un simple éditeur de texte et compiler ensuite votre ``code'' pour produire le document PDF.
    
  \subsection{Les balises}
    \LaTeX\ utilise des balises permettant de préciser comment votre texte doit être compris par l'éditeur.  Ces balises commencent en général par un backslash (\textbackslash).  Il y a deux types de balises les balises qui doivent être ouverte et fermée et les simples balises.  Par exemple, pour définir le contenu de votre document vous devez le mettre entre les balises \verb|\begin{document}| et \verb|\end{document}|.  En fait vous spécifié le début et la fin.  Ces balises ``begin'' et ``end'' permettent de définir un environnement.  Vous retrouverez ces balises dans les parties à venir.\\
    Les balises simples sont beaucoup utilisé pour les caractères spéciaux tel que les symboles mathématiques ou les caractères UTF8 non disponible de base sur un clavier.  Par exemple, pour mettre un backslash vous ne pouvez pas taper \textbackslash car il sera interprété comme le début d'une blise.  Vous devez donc taper: \verb|\textbackslash|.
    
  \subsection{Les packages.}
    Par défault, \LaTeX\ intègre des balises (comme en python, en C++ ou en Java).   Cependant, si vous voulez faire des choses plus complèxe, il vous faudra importer des packages (des sortes de librairire).  Les packages vous permettent donc d'utiliser de nouvelles balises.  Les packages doivent être importé en début de document via la balise \verb|\usepackage{nom du package}|.  Sachez que certains packages se trouvent déjà sur votre ordinateur et ne sont juste pas intégré par défault dans \LaTeX\ alors que d'autres doivent être téléchargé en plus.  Internet vous renseignera mieux que nous sur comment installer un package et sur tous les packages existant.
    
  \subsection{Le corps du texte.}
    % a faire
    
  \subsection{Document déjà fait}
    Voici un ``code'' avec déjà tout pour que vous puissiez écrire vos documents.  A noté que le symbole ``\%'' permet de commencer un commentaire.  C'est à dire du texte qui ne sera pas lu par \LaTeX\ et donc simplement ignoré.
    \begin{verbatim}
      \documentclass[a4paper,11pt]{article}
      \usepackage[french]{babel} % Mise en page Français
      \usepackage[utf8]{inputenc} % Mode d'encodage
      \usepackage[T1]{fontenc}
      \usepackage{amsmath}
      \usepackage{amsfonts}
      \usepackage{amssymb}
      \usepackage{hyperref} %Permet d'avoir des liens automatique dans votre table des matières
      \usepackage{graphicx}
      \usepackage[top=2cm, bottom=2cm, left=2cm, right=2cm]{geometry}
      
      \title{Votre Titre}
      \author{L'auteur}
      
      \begin{document} % Début du document

      \maketitle % Création d'un titre automatiquement

      \newpage % Nouvelle page
      \tableofcontents % Table des matières
      \newpage % Nouvelle page
      
      % Votre Texte
      
      \end{document} % Fin du document
    \end{verbatim}


% - Elements basique -----------------------------------------------------------------------------------------------------
\section{Éléments basique}
  \subsection{Type de document}
    % a faire
    
  \subsection{Structure du contenu} % Chapitre, section, sous-section, paragraphe...
    \subsubsection{Titre}
      Vous pouvez créé des séparation, des titres, des chapitres, ...  Bref, plusieurs méthodes existent pour structurer votre document.  En fonction du type de document (voir paragraphe ci dessus) il y a différentes balises à utiliser.  Voici la liste des balises les plus utilisé permettant d'organiser son code (trié par leur ordre d'importance):
      \begin{enumerate}
        \item \verb|\chapter{titre}| Chapitre (disponible dans les documents article)
        \item \verb|\section{titre}| Section
        \item \verb|\subsection{titre}| Sous-section
        \item \verb|\subsubsection{titre}| Sous-sous-section
        % \item \verb|| 
      \end{enumerate}
      Par exemple, ``Éléments basique'' est une \verb|section| et ``Structure du contenu'' est une \verb|subsubsection|.
    
    \subsubsection{Table des matières.}
      La création d'une table des matière sur \LaTeX\ se fait de manière automatique. Celle-ci sera mis en place via l'instruction suivante :
      \begin{verbatim}
        \tableofcontents
      \end{verbatim}
      Celle-ci va reprendre l'ensemble des structures utilisées dans votre script \LaTeX\ (\verb|chapitre|, \verb|section|, \verb|subsection|, \verb|subsubsection|) et les mettre en page.
      
    \subsubsection{Marges.}
       Si vous avez déjà tenter  de compiler votre script \LaTeX\ une fois, vous vous serez surement demander comment réduire la taille de ces imposantes marges. Pour cela rien de plus simple. Il vous suffit d'importer le package \verb|geometry| et d'y inclure la taille de vos marges comme suit :
      \begin{verbatim}
        \usepackage[top=2cm, bottom=2cm, left=2cm, right=2cm]{geometry}
      \end{verbatim}
    
    \subsubsection{Paragraphe}
      % a faire (Denis)
      
  \subsection{Mise en forme}
    Il est toujours possible de mettre en page le texte dans vos paragraphes.  Notez que la mise en page se fait automatiquement pour les titres.
    \begin{center}
	\begin{tabular}{|l|l|l|}
	  \hline
	  \verb|\textit{texte}| & \textit{I} & Italique\\\hline
	  \verb|\underline{texte}| & \underline{U} & Souligne\\\hline
	  \verb|\textbftexte}| & \textbf{G} & Gras\\\hline
	\end{tabular}
    \end{center}
    Il est également possible d'aligner votre texte au centre en le mettant entre des balises \verb|\begin{center}| et \verb|\end{center}|.
      

% - Liste ----------------------------------------------------------------------------------------------------------------
\section{Liste}
  \subsection{Introduction}
    \LaTeX\ intègre plusieurs type de listes.  Nous vous en présenterons deux dans ce document.
    
  \subsection{Liste numéroté}
    Il s'agit simplement d'une liste avec des numéros devant chaque point.\\
    \begin{minipage}{0.5\textwidth}
     \begin{verbatim}
      \begin{enumerate}
        \item Premier point
        \item Second point
      \end{enumerate}
     \end{verbatim}
    \end{minipage}
    \begin{minipage}{0.5\textwidth}
      \begin{enumerate}
      \item Premier point
      \item Second point
      \end{enumerate}
    \end{minipage}
    
  \subsection{Liste à point}
    Encore plus basique, il s'agit d'une liste avec simplement des points.\\
    \begin{minipage}{0.5\textwidth}
     \begin{verbatim}
      \begin{itemize}
        \item Premier point
        \item Second point
      \end{itemize}
     \end{verbatim}
    \end{minipage}
    \begin{minipage}{0.5\textwidth}
      \begin{itemize}
      \item Premier point
      \item Second point
      \end{itemize}
    \end{minipage}
  
  \subsection{Personnalisation}
    Il vous est possible de personnalisé le caractères mis au début de chaque nouveau point. Pour cela il vous suffit de mettre entre crochet (\verb|[ ]|) le caractères que vous voulez.  Par exemple \textbullet\ ou un +\\
    \begin{minipage}{0.5\textwidth}
     \begin{verbatim}
      \begin{itemize}
        \item[\textbullet] Premier point
        \item[+] Second point
        \item[$-$] Troisième point
      \end{itemize}
     \end{verbatim}
    \end{minipage}
    \begin{minipage}{0.5\textwidth}
      \begin{itemize}
      \item[\textbullet] Premier point
      \item[+] Second point
      \item[$-$] Troisième point
      \end{itemize}
    \end{minipage}

  
% - Tableau --------------------------------------------------------------------------------------------------------------
\section{Tableau}
  \subsection{Introduction}
    Faire un tableau en \LaTeX\ n'est pas toujours simple.  Nous vous expliquons ici les balises que nous utilisons le plus souvent.  Sachez cependant que pour des tableau complexe le plus simple reste d'utiliser un outil disponible sur internet à l'adresse: \url{http://tablesgenerator.com/latex_tables}
  \subsection{Créer un tableau}
    % a faire
  \subsection{Bordure}
    % a faire

% - Image ----------------------------------------------------------------------------------------------------------------
\section{Image}
  % a faire


% - Math -----------------------------------------------------------------------------------------------------------------
\section{Math}
  \subsection{Introduction}
    Il existe plusieurs manière d'écire des formules mathématique.  Quoi qu'il en soit il vous faut rentrer dans un environnement math.  Il existe plusieurs environnement mathématique et plusieurs moyen d'y rentrer.
    
    \subsubsection{Code sur une ligne}
      Pour écrire vos formules mathématique dans une phrase vous devez les mettres \textbf{entre} des balises dollard (\$) ou entre la suite de caractères ``\verb|\(|'' et ``\verb|\)|'' pour fermer la balise. Voici un petit exemple: $\forall a \in \mathbb{R}, \Sigma_{i=1}^{10} (a+i)$
    
    \subsubsection{Code centre}
      A d'autre moment, vous aurez plus envie de mettre en évidence une formule et de la mettre au centre spéciallement sur une ligne.  Pour cela vous devez mettre vos balises \textbf{entre} des doubles dollard (\$\$) soit via la balise ``\verb|\[|'' et ``\verb|\]|''
      \[\forall a \in \mathbb{R}, \Sigma_{i=1}^{10} (a+i)\]
  
  \subsection{Résumé des balises}
    \begin{table}[h]
    \begin{tabular}{|l|l|l|}
      \hline
      \textbf{Balise} & \textbf{Symbole} &\textbf{Description}\\\hline 
      \verb|\forall| & \multicolumn{1}{c|}{$\forall$} &Pour tout\\\hline
      \verb|\in| & \multicolumn{1}{c|}{$\in$} &Appartient à\\\hline
      \verb|\mathbb{Lettre}| & \multicolumn{1}{c|}{$\mathbb{N}$} &Ensemble (remplacer ``Lettre'' par la lettre de l'ensemble voulu)\\\hline
      \verb|\neq| & \multicolumn{1}{c|}{$\neq$} &Différent de..\\\hline
      \verb|\subset| & \multicolumn{1}{c|}{$\subset$} &Inclus dans...\\\hline
      \verb|\supset| & \multicolumn{1}{c|}{$\supset$} &Contient...\\\hline
      \verb|\subseteq| & \multicolumn{1}{c|}{$\subseteq$} &Inclus ou égal à...\\\hline
      \verb|\supseteq| & \multicolumn{1}{c|}{$\supseteq$} &Contient ou égal à...\\\hline
      \verb|\exists| & \multicolumn{1}{c|}{$\exists$} &Il existe ...\\\hline
      \verb|\leq| & \multicolumn{1}{c|}{$\leq$} &Plus petit ou égal\\\hline
      \verb|\geq| & \multicolumn{1}{c|}{$\geq$} &Plus grand ou égal\\\hline
      \verb|\Sigma| & \multicolumn{1}{c|}{$\Sigma$} &Somme\\\hline
      \verb|\log| & \multicolumn{1}{c|}{$\log$} &Logarithme \\\hline
      \verb|\int| & \multicolumn{1}{c|}{$\int$} &Intégral\\\hline
      \verb|\infty| & \multicolumn{1}{c|}{$\infty$} &Infini\\\hline
      \verb|\phi| & \multicolumn{1}{c|}{$\phi$} &Vide\\\hline
      \verb|\sqrt{valeur}| & \multicolumn{1}{c|}{$\sqrt{valeur}$} &Racine de valeur\\\hline
      \verb|\frac{x}{y}| & \multicolumn{1}{c|}{$\frac{x}{y}$} &Fraction (division) de x par y\\\hline
      % \verb| | & \multicolumn{1}{c|}{ } & \\\hline
    \end{tabular}
    \end{table}
    
  \subsection{Lettres grec}
    Pour utiliser les lettres grec il vous suffit de mettre leur nom.  Par exemple $\alpha$ (\verb|\alpha|), $\beta$ (\verb|\beta|), ...  Certaines lettres grecs peuvent être écrite en majuscule comme par exemple sigma.  Pour se faire, il suffit de mettre la première lettre de la commande en majuscule.  Exemple: $\Sigma$ (\verb|\Sigma|) (à la place $\sigma$ (\verb|\sigma|)).
    
  \subsection{Graphes}
    \subsubsection{Exemple}
      \begin{minipage}{\textwidth}
	\center  
	\begin{tikzpicture}
		\draw (0,2) node[anchor=south east]{$a$} node{\textbullet};
		\draw (0,1) node[anchor=east]{$b$} node{\textbullet};
		\draw (0,0) node[anchor=north east]{$c$} node{\textbullet};
		\draw (3,1) node[anchor=west]{$d$} node{\textbullet};
		\draw (0,0.5) ellipse (0.2 and 0.5);
		\draw (0,1.5) ellipse (0.2 and 0.5);
		\draw[color=red] (0,2) -- (3,1);
		\draw  (1.5,1.7) node[color=red] {$e_1$};
		\draw (0,1) -- (3,1);
		\draw  (1.5,0.25) node[color=blue] {$e_2$};
		\draw[color=blue] (0,0) -- (3,1);
	\end{tikzpicture} \\
	\end{minipage}\hfill
    
    \subsubsection{Introduction}
      Nous ne présenterons ici qu'une toute petite partie de ce module simplement pour vous expliqué que cela existe et pour vous expliquer rapidement.  Sachez également qu'il est possible de faire des repères cartésien, ...  Nous vous invitons à consulter internet pour plus d'information.
      
    \subsubsection{Créé l'environnement}
      \begin{verbatim}
        \begin{minipage}{\textwidth}
        \center
        \begin{tikzpicture}
      \end{verbatim}
      Avec ces 4 balises vous serez dans un environnement de dessin.  La seule chose que je vous propose de modifier ici est \verb|{\textwidth}|.  Cela défini la taille que va prendre votre dessin.  Si vous voulez par exemple que le graphique prenne la moitier de la page (en largeur) mettez: \verb|{0.5\textwidth}|.
      
    \subsubsection{Créer un point}
      Exemple: \verb|\draw (0,2) node[anchor=south east]{$a$} node{\textbullet};|\\
      Passsons rapidement en revue cette instruction (l'ordre des nodes n'a pas d'importance).  Nous avons donc l'instruction \verb|\draw| suivit des coorodonnées où placer le point.\\ Vient ensuite \verb|node[anchor=south east]{$a$}| qui décrit le label à placé à coté du point.  Ici on précise (via \verb|anchor| que le point sera placé au sud est du label.\\
      Enfin, on dit de placer un \verb|\textbullet| à cette cordonnées (ce qui correspond à \textbullet).
      
    \subsection{Créer une ligne}
      Exemple: \verb|\draw (0,1) -- (3,1);|\\
      On retrouve de nouveau les coordonnées.  La syntaxe est assez simple, on a les 2 coordonnées à relier lié par 2 tirrets représentant la ligne.  Ces deux tirrets peuvent être changé par ellipse ou autre (la syntaxe des coordonnées deva peut-être un peu changé ou précisé).
      
  \subsection{Matrice} 
    \subsubsection{Introduction}
      Nous parlerons ici de seulement deux sortes de matrices: ( ) et | |.\\
      
    \subsubsection{Matrice classique}
      Pour faire une matrice, il vous faut rentrer dans l'invironnement math (avec des \$ par exemple).  Ensuite vous devez spécifié que vous voulez une matrice, ici nous utiliseront ``pmatrix''.  Nous allons donc faire \verb|$\begin{pmatrix}|.  Ensuite, cela va se passer comment un tableau.  C'est à dire un ``\&'' pour séparré le colonne et ``\textbackslash\textbackslash'' pour les lignes.  Cela donne donc:\\                                                                                                                                                                                                                       
      \begin{minipage}{0.5\textwidth}
        \begin{verbatim}
         $\begin{pmatrix}
         a & b\\
         c & d
         \end{pmatrix}$
        \end{verbatim}
      \end{minipage}
      \begin{minipage}{0.5\textwidth}
       $\begin{pmatrix}
	a & b\\
	c & d
       \end{pmatrix}$
      \end{minipage}

    \subsubsection{Déterminant}
      Il vous suffit de suivre la même syntaxe pour faire des matrices déterminante. Un bonne exemple vaut mieux que des explications:\\
      \begin{minipage}{0.5\textwidth}
        \begin{verbatim}
         $\begin{vmatrix}
         a & b\\
         c & d
         \end{vmatrix}$
        \end{verbatim}
      \end{minipage}
      \begin{minipage}{0.5\textwidth}
       $\begin{vmatrix}
	a & b\\
	c & d
       \end{vmatrix}$
      \end{minipage}
      
    
  
\end{document}
